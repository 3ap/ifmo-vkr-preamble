% Pandoc-specific для Markdown
\def\tightlist{}
\usepackage{multirow}
\usepackage{longtable}
\usepackage{booktabs}

% Поиск и копипаст по pdf
\usepackage{cmap}

% Использование графики в pdf
\usepackage[pdftex]{graphicx}

% Поддержка гиперссылок внутри pdf
\usepackage[pdftex]{hyperref}
\hypersetup{
        unicode=true,
        pdftitle={
        },
        pdfauthor={},
        pdfkeywords={
        },
        colorlinks,
        citecolor=black,
        filecolor=black,
        linkcolor=black,
        urlcolor=blue
}

% Разрешить перенос слов в URL'ах после любой буквы, так мы обеспечим
% правильные поля в списке источников и везде, где есть ссылки
\expandafter\def\expandafter\UrlBreaks\expandafter{\UrlBreaks%  save the current one
  \do\a\do\b\do\c\do\d\do\e\do\f\do\g\do\h\do\i\do\j%
  \do\k\do\l\do\m\do\n\do\o\do\p\do\q\do\r\do\s\do\t%
  \do\u\do\v\do\w\do\x\do\y\do\z\do\A\do\B\do\C\do\D%
  \do\E\do\F\do\G\do\H\do\I\do\J\do\K\do\L\do\M\do\N%
  \do\O\do\P\do\Q\do\R\do\S\do\T\do\U\do\V\do\W\do\X%
  \do\Y\do\Z}

% Вставка листингов кода
\usepackage{listings}
\newcommand{\includecode}[3]{\lstinputlisting[caption=#3, escapechar=, style=custom#1]{#2}}

% Экзотические цвета (оранжевый, фиолетовый и др.)
\usepackage{xcolor}

% Кастомный стиль подсветки для языка Си
\lstdefinestyle{customc}{
  belowcaptionskip=1\baselineskip,
  breaklines=true,
  frame=none,
  xleftmargin=\parindent,
  language=C,
  showstringspaces=false,
  basicstyle=\normalsize,
  keywordstyle=\bfseries\color{green!40!black},
  commentstyle=\itshape\color{purple!40!black},
  identifierstyle=\color{black},
  stringstyle=\color{orange!40!black},
}

% Сдвиги для списков
\usepackage{enumitem}
\setlist[enumerate]{topsep=0pt,itemsep=0ex,partopsep=1ex,parsep=1ex}
\setlist[itemize]{itemsep=0ex}

% Красивый маркер ненумерованного списка в виде тире
\def\labelitemi{--}

% Просто адекватные отступы (не по ГОСТу)
\titlespacing*{\subsection}   {0pt}{\baselineskip}{\baselineskip}
\titlespacing*{\subsubsection}{0pt}{\baselineskip}{\baselineskip}

% Поддержка SVG
\usepackage{svg}

% Возможность повернуть любую из страниц
\usepackage{pdflscape}

% Возможность описывать байтовые структуры
\usepackage{bytefield}
