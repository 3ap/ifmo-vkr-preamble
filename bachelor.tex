% Документ, по которому в основном мы ориентировались при написании
% этой преамбулы:
%
%   http://edu.ifmo.ru/file/pages/14/trebovaniya_k_vypusknym_kvalifikacionnym_rabotam.pdf
%

% Times New Roman для русского языка
%   Необходим установленный пакет pscyr
\usepackage{pscyr}
\renewcommand{\rmdefault}{ftm}

% Полуторный межстрочный интервал
\usepackage[nodisplayskipstretch]{setspace}
\onehalfspacing

% Правильные поля для диплома
\usepackage[top=20mm, bottom=20mm, left=25mm, right=10mm]{geometry}

\addto\captionsrussian{
% подпись "Рисунок" вместо "Рис"
  \def\figurename{{Рисунок}}
% "Оглавление" вместо "Содержание"
  \def\figurename{{Оглавление}}
}

% Каждый пункт оглавления должен быть с отточием
\usepackage{titletoc}

% Максимальная вложенность содержания (только разделы, подразделы и
% "пункты")
\setcounter{tocdepth}{3}

% Оглавление должно начинаться на 4 странице
\setcounter{page}{4}

% Возможность переопределять оглавление и его стиль
\usepackage{tocloft}

\usepackage{etoolbox}

% Слово "Оглавление" заглавными буквами
\makeatletter
\patchcmd{\@cftmaketoctitle}{\cfttoctitlefont\contentsname}{\cfttoctitlefont\MakeUppercase{\contentsname}}{}{}
\makeatother

% Абзацный отступ равен 1.25 см
\parindent=1.25cm

% Возможность менять регистр текста в UTF-8
\usepackage{textcase}

% Формат заголовков
%   - Заголовок раздела по центру, кернингом побольше (отсебятина),
%     прописными буквами, выделено жирным, X   ЗАГОЛОВОК
%   - Заголовок подраздела и "пункта" со смещением, как у абзаца, по
%     левому краю, выделено жирным, X.Y[.Z]   Заголовок
\usepackage{titlesec}
\titleformat{\section}[block]{\centering\bfseries\large}
                         {\arabic{section}}{1ex}{\MakeUppercase}
\titleformat{\subsection}[block]{\hspace{\parindent}\bfseries\normalsize}
                         {\arabic{section}.\arabic{subsection}}{1ex}{}
\titleformat{\subsubsection}[block]{\hspace{\parindent}\bfseries\normalsize}
                         {\arabic{section}.\arabic{subsection}.\arabic{subsubsection}}{1ex}{}

% TODO: 14pt * 3 = 42pt (три интервала до и после)
\titlespacing*{\section}      {0pt}{42pt}{42pt}

% Номер страницы по середине верхнего поля
\usepackage{fancyhdr}
\pagestyle{fancy}
\fancyhf{}
\fancyhead[C]{\thepage}
\renewcommand{\headrulewidth}{0pt}

% Если источников несколько, то сжать из [1,2,3,4] в [1-4]
\usepackage[numbers,sort&compress]{natbib}

% Стиль нумерования в списке использованных источников: [1] -> 1.
\makeatletter
\renewcommand\@biblabel[1]{#1.}
\makeatother

% Самое длинное тире в качестве разделителя в подписях к рисункам,
% таблицам, листингам и др.
\usepackage{caption}
\DeclareCaptionLabelSeparator{emdash}{ --- }
\captionsetup{labelsep=emdash}

% Подпись к таблице должна быть по левому краю
\captionsetup[table]{singlelinecheck=false}

% Сквозная нумерация таблиц, формул, рисунков
\renewcommand{\theequation}{\arabic{equation}}
\renewcommand{\thetable}{\arabic{table}}
\renewcommand{\thefigure}{\arabic{figure}}

% Добавить абзацный отступ для первых абзацев в section/subsection,
% по умолчанию не добавляется
\usepackage{indentfirst}

% Возможность вставлять таблицы и рисунки непосредственно там, где
% они определены (аргумент [H]). Нужно, чтобы таблицы и рисунки были
% всегда определены под текстом, где на них ссылаются
\usepackage{float}

% Обязательно переносить слова, чтобы соблюсти поля документа. Для
% соблюдения полей можно пренебречь правилами для тех слов и
% словосочетаний, о которых не знают словаря переносов (ruhyphen или
% ruenhyph). Оно почему-то работает. Взято с:
%
%   http://www.latex-community.org/forum/viewtopic.php?p=70342#p70342
%
\tolerance 1414
\hbadness 1414
\emergencystretch 1.5em
\hfuzz 0.3pt
\widowpenalty=10000
\vfuzz \hfuzz
\raggedbottom
